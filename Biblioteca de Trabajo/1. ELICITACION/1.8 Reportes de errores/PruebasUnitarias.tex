\documentclass[12pt,a4paper]{article}

% ============================================
% PAQUETES
% ============================================
\usepackage[utf8]{inputenc}
\usepackage[T1]{fontenc}
\usepackage[spanish,es-tabla]{babel}
\usepackage{graphicx}
\usepackage{geometry}
\usepackage{fancyhdr}
\usepackage{xcolor}
\usepackage{listings}
\usepackage{amssymb}
\usepackage{amsmath}
\usepackage{booktabs}
\usepackage{longtable}
\usepackage{array}
\usepackage{hyperref}
\usepackage{tcolorbox}
\usepackage{enumitem}
\usepackage{float}
\usepackage{caption}
\usepackage{subcaption}
\usepackage{titlesec}
\usepackage{tocloft}
\usepackage{parskip}
\usepackage{tabularx}
\usepackage{multirow}

% ============================================
% CONFIGURACION DE PAGINA
% ============================================
\geometry{
    left=3cm,
    right=2.5cm,
    top=2.5cm,
    bottom=2.5cm
}
\setlength{\headheight}{14.5pt}

% ============================================
% CONFIGURACION DE ENCABEZADOS
% ============================================
\pagestyle{fancy}
\fancyhf{}
\fancyhead[L]{\small\leftmark}
\fancyhead[R]{\small KairosMix V2.0}
\fancyfoot[C]{\thepage}
\renewcommand{\headrulewidth}{0.5pt}
\renewcommand{\footrulewidth}{0.3pt}

% ============================================
% CONFIGURACION DE INDICE
% ============================================
\renewcommand{\cftsecleader}{\cftdotfill{\cftdotsep}}
\renewcommand{\cftsubsecleader}{\cftdotfill{\cftdotsep}}
\renewcommand{\cftsubsubsecleader}{\cftdotfill{\cftdotsep}}
\setlength{\cftbeforesecskip}{6pt}
\setlength{\cftbeforesubsecskip}{3pt}

% ============================================
% CONFIGURACION DE SECCIONES
% ============================================
\titleformat{\section}
    {\normalfont\Large\bfseries}{\thesection.}{1em}{}
\titleformat{\subsection}
    {\normalfont\large\bfseries}{\thesubsection.}{1em}{}
\titleformat{\subsubsection}
    {\normalfont\normalsize\bfseries}{\thesubsubsection.}{1em}{}

% ============================================
% COLORES PERSONALIZADOS
% ============================================
\definecolor{codegreen}{rgb}{0,0.6,0}
\definecolor{codegray}{rgb}{0.5,0.5,0.5}
\definecolor{codepurple}{rgb}{0.58,0,0.82}
\definecolor{backcolour}{rgb}{0.97,0.97,0.95}
\definecolor{passgreen}{RGB}{40,167,69}
\definecolor{failred}{RGB}{220,53,69}
\definecolor{primaryblue}{RGB}{0,82,147}
\definecolor{secondaryblue}{RGB}{23,162,184}

% ============================================
% CONFIGURACION DE LISTINGS
% ============================================
% Definicion del lenguaje JavaScript para listings
\lstdefinelanguage{JavaScript}{
    keywords={break, case, catch, continue, debugger, default, delete, do, else, false, finally, for, function, if, in, instanceof, new, null, return, switch, this, throw, true, try, typeof, var, void, while, with, let, const, class, export, extends, import, super, async, await, static, get, set, of},
    morecomment=[l]{//},
    morecomment=[s]{/*}{*/},
    morestring=[b]',
    morestring=[b]",
    morestring=[b]`,
    sensitive=true
}

\lstdefinestyle{codestyle}{
    backgroundcolor=\color{backcolour},
    commentstyle=\color{codegreen}\itshape,
    keywordstyle=\color{codepurple}\bfseries,
    numberstyle=\tiny\color{codegray},
    stringstyle=\color{codegreen},
    basicstyle=\ttfamily\small,
    breakatwhitespace=false,
    breaklines=true,
    captionpos=b,
    keepspaces=true,
    numbers=left,
    numbersep=8pt,
    showspaces=false,
    showstringspaces=false,
    showtabs=false,
    tabsize=2,
    frame=single,
    framesep=3pt,
    rulecolor=\color{codegray},
    xleftmargin=15pt
}
\lstset{style=codestyle}

% ============================================
% CONFIGURACION DE HYPERREF
% ============================================
\hypersetup{
    colorlinks=true,
    linkcolor=primaryblue,
    filecolor=magenta,
    urlcolor=secondaryblue,
    citecolor=primaryblue,
    pdftitle={Informe de Pruebas Unitarias - Sistema KairosMix V2.0},
    pdfauthor={Grupo 6 - Analisis y Diseno de Sistemas},
    pdfkeywords={testing, unit tests, React, Vitest, quality assurance}
}

% ============================================
% CAJAS DE INFORMACION
% ============================================
\tcbuselibrary{skins,breakable}

\newtcolorbox{notabox}[1][]{
    enhanced,
    colback=blue!5!white,
    colframe=primaryblue,
    fonttitle=\bfseries,
    title=#1,
    breakable,
    boxrule=0.5pt,
    left=8pt,
    right=8pt
}

\newtcolorbox{resultbox}[1][]{
    enhanced,
    colback=green!5!white,
    colframe=passgreen,
    fonttitle=\bfseries,
    title=#1,
    breakable,
    boxrule=0.5pt,
    left=8pt,
    right=8pt
}

\newtcolorbox{alertbox}[1][]{
    enhanced,
    colback=yellow!8!white,
    colframe=orange!80!black,
    fonttitle=\bfseries,
    title=#1,
    breakable,
    boxrule=0.5pt,
    left=8pt,
    right=8pt
}

% ============================================
% COMANDOS PERSONALIZADOS
% ============================================
\newcommand{\codigofuente}[1]{\texttt{\small #1}}
\newcommand{\archivo}[1]{\texttt{#1}}
\newcommand{\comando}[1]{\colorbox{backcolour}{\texttt{\small #1}}}

% ============================================
% INICIO DEL DOCUMENTO
% ============================================
\begin{document}

% ============================================
% PORTADA
% ============================================
\begin{titlepage}
    \centering
    \vspace*{1cm}
    
    {\scshape\Large Universidad de las Fuerzas Armadas ''ESPE''\par}
    {\scshape\large Departamento de Ciencias de la Computacion\par}
    {\scshape\normalsize Ingenieria de Software\par}
    
    \vspace{1.5cm}
    
    \rule{\textwidth}{1.5pt}
    \vspace{0.4cm}
    {\Huge\bfseries INFORME DE PRUEBAS UNITARIAS\par}
    \vspace{0.3cm}
    {\Large\bfseries Sistema KairosMix V2.0\par}
    \vspace{0.2cm}
    {\large Sistema de Gestion de Comercializacion de Frutos Secos\par}
    \rule{\textwidth}{1.5pt}
    
    \vspace{1.5cm}
    
    \begin{tabular}{rl}
        \textbf{Asignatura:} & Analisis y Diseno de Sistemas \\
        \textbf{Codigo:} & 27837\_G6\_ADS \\
        \textbf{Grupo:} & 6 \\
        \textbf{Periodo:} & Semestre VI - 2025-2026 \\
    \end{tabular}
    
    \vspace{1.5cm}
    
    \begin{tabular}{|l|}
        \hline
        \textbf{Integrantes del Equipo} \\
        \hline
        % Agregar nombres de integrantes aqui
        Caetano Flores \\
        Jordan Guaman \\
        Anthony Morales \\
        Leonardo Narvaez \\
        \hline
    \end{tabular}
    
    \vfill
    
    {\large\bfseries Enero 2026\par}
\end{titlepage}

% ============================================
% PAGINA DE CONTROL DE VERSIONES
% ============================================
\thispagestyle{empty}
\section*{Control de Versiones del Documento}

\begin{table}[H]
\centering
\begin{tabular}{|c|c|p{6cm}|c|}
\hline
\textbf{Version} & \textbf{Fecha} & \textbf{Descripcion} & \textbf{Autor} \\
\hline
2.0 & 17/01/2026 & Creacion inicial del documento de pruebas unitarias & Grupo 6 \\
\hline
% Agregar mas versiones segun sea necesario
\end{tabular}
\caption{Historial de versiones}
\end{table}

\vspace{1cm}

\section*{Aprobaciones}

\begin{table}[H]
\centering
\begin{tabular}{|l|c|c|c|}
\hline
\textbf{Rol} & \textbf{Nombre} & \textbf{Firma} & \textbf{Fecha} \\
\hline
Elaborado por: & & & \\
\hline
Revisado por: & & & \\
\hline
Aprobado por: & & & \\
\hline
\end{tabular}
\caption{Registro de aprobaciones}
\end{table}

\newpage

% ============================================
% INDICE
% ============================================
\tableofcontents
\newpage

\listoftables
\listoffigures
\newpage

% ============================================
% SECCION 1: INTRODUCCION
% ============================================
\section{Introduccion}

\subsection{Proposito del Documento}

El presente documento constituye el informe tecnico de las pruebas unitarias implementadas para el sistema \textbf{KairosMix V2.0}, desarrollado como parte del proyecto de la asignatura Analisis y Diseno de Sistemas. Este informe tiene como objetivo documentar de manera exhaustiva el proceso de aseguramiento de calidad del software mediante pruebas automatizadas.

\subsection{Objetivos}

\subsubsection{Objetivo General}
Verificar y validar el correcto funcionamiento de los componentes individuales del sistema KairosMix V2.0 mediante la implementacion de pruebas unitarias automatizadas.

\subsubsection{Objetivos Especificos}
\begin{enumerate}[label=\arabic*.]
    \item Implementar pruebas unitarias para cada modulo funcional del sistema.
    \item Validar las reglas de negocio implementadas en los componentes.
    \item Verificar el correcto funcionamiento de las validaciones de entrada de datos.
    \item Documentar los casos de prueba y sus resultados.
    \item Garantizar una cobertura minima del 100\% en funcionalidades criticas.
\end{enumerate}

\subsection{Alcance}

Las pruebas unitarias desarrolladas cubren los siguientes modulos del sistema:

\begin{table}[H]
\centering
\caption{Matriz de cobertura de pruebas por modulo}
\label{tab:cobertura}
\begin{tabular}{|l|l|c|c|}
\hline
\textbf{Modulo} & \textbf{Componentes} & \textbf{N. Tests} & \textbf{Cobertura} \\
\hline
Gestion de Productos & ProductManager, ProductForm & 22 & 100\% \\
\hline
Gestion de Clientes & ClientManager, ClientForm & 34 & 100\% \\
\hline
Gestion de Pedidos & OrderManager, OrderForm & 35 & 100\% \\
\hline
Mezclas Personalizadas & CustomMixDesigner & 32 & 100\% \\
\hline
Funciones Utilitarias & utils.js & 33 & 100\% \\
\hline
Datos de Inicializacion & seedData.js & 20 & 100\% \\
\hline
\multicolumn{2}{|r|}{\textbf{TOTAL}} & \textbf{176} & \textbf{100\%} \\
\hline
\end{tabular}
\end{table}

\subsection{Definiciones, Acronimos y Abreviaturas}

\begin{table}[H]
\centering
\caption{Glosario de terminos tecnicos}
\label{tab:glosario}
\begin{tabular}{|l|p{10cm}|}
\hline
\textbf{Termino} & \textbf{Definicion} \\
\hline
Prueba Unitaria & Verificacion automatizada del comportamiento de una unidad aislada de codigo \\
\hline
Test Suite & Conjunto de casos de prueba agrupados logicamente \\
\hline
Mock & Objeto simulado que replica el comportamiento de dependencias externas \\
\hline
Assertion & Declaracion que verifica si una condicion es verdadera \\
\hline
Coverage & Porcentaje de codigo fuente ejecutado durante las pruebas \\
\hline
TDD & Test-Driven Development (Desarrollo Guiado por Pruebas) \\
\hline
DOM & Document Object Model (Modelo de Objetos del Documento) \\
\hline
jsdom & Implementacion de DOM para entornos Node.js \\
\hline
\end{tabular}
\end{table}

\newpage

% ============================================
% SECCION 2: MARCO TEORICO
% ============================================
\section{Marco Teorico}

\subsection{Pruebas Unitarias en el Desarrollo de Software}

Las pruebas unitarias constituyen el nivel mas bajo de la piramide de testing y son fundamentales para garantizar la calidad del software. Segun Martin Fowler, las pruebas unitarias deben ser:

\begin{itemize}
    \item \textbf{Rapidas:} Ejecutarse en milisegundos para permitir ejecucion frecuente.
    \item \textbf{Aisladas:} No depender de sistemas externos ni de otras pruebas.
    \item \textbf{Repetibles:} Producir el mismo resultado en cada ejecucion.
    \item \textbf{Auto-verificables:} Determinar automaticamente si pasaron o fallaron.
    \item \textbf{Oportunas:} Escribirse antes o junto con el codigo de produccion.
\end{itemize}

\subsection{Metodologia de Testing Utilizada}

Para el desarrollo de las pruebas unitarias del sistema KairosMix V2.0 se aplico la metodologia \textbf{Arrange-Act-Assert (AAA)}, que estructura cada caso de prueba en tres fases:

\begin{enumerate}
    \item \textbf{Arrange (Preparar):} Configurar el estado inicial y los datos de entrada.
    \item \textbf{Act (Actuar):} Ejecutar la funcionalidad bajo prueba.
    \item \textbf{Assert (Verificar):} Comprobar que el resultado es el esperado.
\end{enumerate}

\subsection{Herramientas y Tecnologias}

\begin{table}[H]
\centering
\caption{Stack tecnologico de testing}
\label{tab:tecnologias}
\begin{tabular}{|l|l|p{7cm}|}
\hline
\textbf{Herramienta} & \textbf{Version} & \textbf{Proposito} \\
\hline
Vitest & 4.0.17 & Framework de testing nativo para Vite, compatible con Jest \\
\hline
React Testing Library & Latest & Testing de componentes React con enfoque en accesibilidad \\
\hline
jest-dom & Latest & Matchers personalizados para aserciones de DOM \\
\hline
user-event & Latest & Simulacion realista de interacciones de usuario \\
\hline
jsdom & Latest & Implementacion de DOM para Node.js \\
\hline
\end{tabular}
\end{table}

\newpage

% ============================================
% SECCION 3: CONFIGURACION DEL ENTORNO
% ============================================
\section{Configuracion del Entorno de Pruebas}

\subsection{Arquitectura de Testing}

El entorno de pruebas se estructura de la siguiente manera:

\begin{verbatim}
src/
  test/
    setup.js              (Configuracion global de tests)
    testUtils.jsx         (Utilidades y datos mock)
  components/
    Products/
      ProductManager.test.jsx
    Clients/
      ClientManager.test.jsx
    Orders/
      OrderManager.test.jsx
    CustomMix/
      CustomMixDesigner.test.jsx
  utils/
    utils.test.js
  data/
    seedData.test.js
\end{verbatim}

\subsection{Configuracion de Vitest}

El archivo \archivo{vitest.config.js} define la configuracion del framework de testing:

\begin{lstlisting}[language=JavaScript, caption=Configuracion de Vitest (vitest.config.js)]
import { defineConfig } from 'vitest/config'
import react from '@vitejs/plugin-react'

export default defineConfig({
  plugins: [react()],
  test: {
    globals: true,
    environment: 'jsdom',
    setupFiles: './src/test/setup.js',
    css: true,
    coverage: {
      provider: 'v8',
      reporter: ['text', 'json', 'html'],
      exclude: [
        'node_modules/',
        'src/test/',
        '**/*.config.js',
        'dist/'
      ]
    },
    include: ['src/**/*.{test,spec}.{js,jsx}'],
    testTimeout: 10000
  }
})
\end{lstlisting}

\subsection{Configuracion Global de Tests}

El archivo \archivo{setup.js} inicializa el entorno con los mocks necesarios:

\begin{lstlisting}[language=JavaScript, caption=Configuracion global (src/test/setup.js)]
import { expect, afterEach, vi, beforeEach } from 'vitest'
import { cleanup } from '@testing-library/react'
import * as matchers from '@testing-library/jest-dom/matchers'

// Extender expect con matchers de jest-dom
expect.extend(matchers)

// Limpiar despues de cada test
afterEach(() => {
  cleanup()
})

// Mock de localStorage
const localStorageMock = {
  getItem: vi.fn(),
  setItem: vi.fn(),
  removeItem: vi.fn(),
  clear: vi.fn(),
}

Object.defineProperty(window, 'localStorage', {
  value: localStorageMock,
})

// Mock de SweetAlert2
vi.mock('sweetalert2', () => ({
  default: {
    fire: vi.fn(() => Promise.resolve({ isConfirmed: true })),
    close: vi.fn(),
  }
}))
\end{lstlisting}

\newpage

% ============================================
% SECCION 4: ESPECIFICACION DE CASOS DE PRUEBA
% ============================================
\section{Especificacion de Casos de Prueba}

\subsection{Modulo de Gestion de Productos}

\subsubsection{Descripcion Funcional}
El modulo de gestion de productos permite realizar operaciones CRUD sobre el inventario de frutos secos del sistema.

\subsubsection{Casos de Prueba}

\begin{table}[H]
\centering
\caption{Casos de prueba - Formulario de Productos}
\label{tab:cp-productos}
\begin{tabular}{|c|p{6cm}|c|c|}
\hline
\textbf{ID} & \textbf{Descripcion} & \textbf{Tipo} & \textbf{Estado} \\
\hline
CP-P01 & Renderizado inicial del formulario vacio & Funcional & $\checkmark$ \\
\hline
CP-P02 & Carga de datos en modo edicion & Funcional & $\checkmark$ \\
\hline
CP-P03 & Validacion de nombre obligatorio & Validacion & $\checkmark$ \\
\hline
CP-P04 & Validacion de rango de precios (0.01-99.99) & Validacion & $\checkmark$ \\
\hline
CP-P05 & Validacion de stock entero positivo & Validacion & $\checkmark$ \\
\hline
CP-P06 & Generacion automatica de codigo & Logica & $\checkmark$ \\
\hline
CP-P07 & Incremento de codigo existente & Logica & $\checkmark$ \\
\hline
CP-P08 & Validacion de formato de imagen & Validacion & $\checkmark$ \\
\hline
CP-P09 & Validacion de tamano de imagen (max 5MB) & Validacion & $\checkmark$ \\
\hline
CP-P10 & Ejecucion de callback onCancel & Funcional & $\checkmark$ \\
\hline
\end{tabular}
\end{table}

\subsubsection{Codigo de Prueba - Validacion de Precios}

\begin{lstlisting}[language=JavaScript, caption=Test de validacion de precios]
describe('Validacion de precios', () => {
  const validatePrice = (price) => {
    const priceRegex = /^\d{1,2}(\.\d{1,2})?$/
    const numValue = parseFloat(price)
    return priceRegex.test(price) && 
           numValue >= 0.01 && 
           numValue <= 99.99
  }

  it('debe aceptar precios en rango valido', () => {
    expect(validatePrice('15.99')).toBe(true)
    expect(validatePrice('0.01')).toBe(true)
    expect(validatePrice('99.99')).toBe(true)
  })

  it('debe rechazar precios fuera de rango', () => {
    expect(validatePrice('100.00')).toBe(false)
    expect(validatePrice('0')).toBe(false)
    expect(validatePrice('-5')).toBe(false)
  })
})
\end{lstlisting}

\subsubsection{Codigo de Prueba - Generacion de Codigos}

\begin{lstlisting}[language=JavaScript, caption=Test de generacion de codigos de producto]
describe('Generacion de codigos de producto', () => {
  const generateProductCode = (productName, existingProducts) => {
    if (!productName.trim()) return ''
    const firstLetter = productName.trim().charAt(0).toUpperCase()
    const existingCodes = existingProducts
      .filter(p => p.code && p.code.startsWith(firstLetter))
      .map(p => p.code).sort()
    
    for (let i = 1; i <= 20; i++) {
      const newCode = `${firstLetter}${i.toString().padStart(2, '0')}`
      if (!existingCodes.includes(newCode)) return newCode
    }
    return `${firstLetter}21`
  }

  it('debe generar codigo A01 para "Almendras"', () => {
    expect(generateProductCode('Almendras', [])).toBe('A01')
  })

  it('debe incrementar numero cuando codigo existe', () => {
    const existing = [{ code: 'A01' }, { code: 'A02' }]
    expect(generateProductCode('Avellanas', existing)).toBe('A03')
  })
})
\end{lstlisting}

\newpage

\subsection{Modulo de Gestion de Clientes}

\subsubsection{Descripcion Funcional}
El modulo gestiona la informacion de los clientes, incluyendo validaciones especificas para documentos de identidad ecuatorianos.

\subsubsection{Casos de Prueba}

\begin{table}[H]
\centering
\caption{Casos de prueba - Validacion de Identificacion}
\label{tab:cp-clientes-id}
\begin{tabular}{|c|p{5cm}|l|c|}
\hline
\textbf{ID} & \textbf{Descripcion} & \textbf{Entrada} & \textbf{Estado} \\
\hline
CP-C01 & Cedula valida de 10 digitos & 1234567890 & $\checkmark$ \\
\hline
CP-C02 & Cedula con menos de 10 digitos & 123456789 & $\checkmark$ \\
\hline
CP-C03 & Cedula con mas de 10 digitos & 12345678901 & $\checkmark$ \\
\hline
CP-C04 & Cedula con caracteres alfabeticos & 123456789A & $\checkmark$ \\
\hline
CP-C05 & RUC valido de 13 digitos & 1234567890001 & $\checkmark$ \\
\hline
CP-C06 & RUC con menos de 13 digitos & 123456789000 & $\checkmark$ \\
\hline
CP-C07 & Pasaporte valido (6-9 caracteres) & AB123456 & $\checkmark$ \\
\hline
CP-C08 & Pasaporte muy corto & AB123 & $\checkmark$ \\
\hline
\end{tabular}
\end{table}

\begin{table}[H]
\centering
\caption{Casos de prueba - Validacion de Contacto}
\label{tab:cp-clientes-contacto}
\begin{tabular}{|c|p{5cm}|l|c|}
\hline
\textbf{ID} & \textbf{Descripcion} & \textbf{Entrada} & \textbf{Estado} \\
\hline
CP-C09 & Email con formato valido & user@domain.com & $\checkmark$ \\
\hline
CP-C10 & Email sin arroba & userdomain.com & $\checkmark$ \\
\hline
CP-C11 & Email sin dominio & user@ & $\checkmark$ \\
\hline
CP-C12 & Email con espacios & user @domain.com & $\checkmark$ \\
\hline
CP-C13 & Telefono movil valido & 0987654321 & $\checkmark$ \\
\hline
CP-C14 & Telefono con formato incorrecto & 987654321 & $\checkmark$ \\
\hline
\end{tabular}
\end{table}

\subsubsection{Codigo de Prueba - Validacion de Identificacion}

\begin{lstlisting}[language=JavaScript, caption=Test de validacion de documentos de identidad]
describe('Validacion de numero de identificacion', () => {
  const validateIdNumber = (idNumber, idType) => {
    const cleanId = idNumber.replace(/\s/g, '').toUpperCase()
    
    switch(idType) {
      case 'cedula':
        return /^\d{10}$/.test(cleanId)
      case 'ruc':
        return /^\d{13}$/.test(cleanId)
      case 'pasaporte':
        return /^[A-Z0-9]{6,9}$/.test(cleanId)
      default:
        return false
    }
  }

  describe('Validacion de Cedula Ecuatoriana', () => {
    it('acepta cedula valida de 10 digitos', () => {
      expect(validateIdNumber('1234567890', 'cedula')).toBe(true)
    })

    it('rechaza cedula con longitud incorrecta', () => {
      expect(validateIdNumber('123456789', 'cedula')).toBe(false)
      expect(validateIdNumber('12345678901', 'cedula')).toBe(false)
    })

    it('rechaza cedula con caracteres no numericos', () => {
      expect(validateIdNumber('123456789A', 'cedula')).toBe(false)
    })
  })
})
\end{lstlisting}

\newpage

\subsection{Modulo de Gestion de Pedidos}

\subsubsection{Descripcion Funcional}
El modulo implementa una maquina de estados finitos para gestionar el ciclo de vida de los pedidos, desde su creacion hasta su completacion o cancelacion.

\subsubsection{Diagrama de Estados}

\begin{notabox}[Maquina de Estados de Pedidos]
El sistema implementa las siguientes transiciones de estado:
\begin{itemize}
    \item \textbf{Pendiente} $\to$ En Proceso, Cancelado
    \item \textbf{En Proceso} $\to$ En Espera, Completado, Cancelado
    \item \textbf{En Espera} $\to$ En Proceso, Cancelado
    \item \textbf{Completado} $\to$ (Estado final - sin transiciones)
    \item \textbf{Cancelado} $\to$ (Estado final - sin transiciones)
\end{itemize}
\end{notabox}

\subsubsection{Casos de Prueba - Transiciones de Estado}

\begin{table}[H]
\centering
\caption{Casos de prueba - Maquina de Estados}
\label{tab:cp-estados}
\begin{tabular}{|c|l|l|c|c|}
\hline
\textbf{ID} & \textbf{Estado Origen} & \textbf{Estado Destino} & \textbf{Valido} & \textbf{Estado} \\
\hline
CP-O01 & Pendiente & En Proceso & Si & $\checkmark$ \\
\hline
CP-O02 & Pendiente & Cancelado & Si & $\checkmark$ \\
\hline
CP-O03 & Pendiente & Completado & No & $\checkmark$ \\
\hline
CP-O04 & En Proceso & Completado & Si & $\checkmark$ \\
\hline
CP-O05 & En Proceso & En Espera & Si & $\checkmark$ \\
\hline
CP-O06 & Completado & Cualquier estado & No & $\checkmark$ \\
\hline
CP-O07 & Cancelado & Cualquier estado & No & $\checkmark$ \\
\hline
\end{tabular}
\end{table}

\subsubsection{Casos de Prueba - Calculo de Totales}

\begin{table}[H]
\centering
\caption{Casos de prueba - Calculos Financieros}
\label{tab:cp-calculos}
\begin{tabular}{|c|p{5cm}|c|c|c|}
\hline
\textbf{ID} & \textbf{Descripcion} & \textbf{Subtotal} & \textbf{IVA (15\%)} & \textbf{Estado} \\
\hline
CP-O08 & Calculo de subtotal simple & \$95.00 & \$14.25 & $\checkmark$ \\
\hline
CP-O09 & Calculo con multiples items & \$250.00 & \$37.50 & $\checkmark$ \\
\hline
CP-O10 & Redondeo a 2 decimales & \$99.99 & \$15.00 & $\checkmark$ \\
\hline
CP-O11 & Lista vacia de productos & \$0.00 & \$0.00 & $\checkmark$ \\
\hline
\end{tabular}
\end{table}

\subsubsection{Codigo de Prueba - Calculo de Totales}

\begin{lstlisting}[language=JavaScript, caption=Test de calculo de totales con IVA]
describe('Calculo de totales de pedido', () => {
  const IVA_RATE = 0.15 // 15% IVA Ecuador
  
  const calculateOrderTotals = (products) => {
    const subtotal = products.reduce((sum, item) => {
      return sum + (item.quantity * item.price)
    }, 0)
    
    const taxes = subtotal * IVA_RATE
    const total = subtotal + taxes
    
    return {
      subtotal: Math.round(subtotal * 100) / 100,
      taxes: Math.round(taxes * 100) / 100,
      total: Math.round(total * 100) / 100
    }
  }

  it('calcula subtotal correctamente', () => {
    const products = [
      { quantity: 5, price: 10.00 },
      { quantity: 3, price: 15.00 }
    ]
    const result = calculateOrderTotals(products)
    expect(result.subtotal).toBe(95.00)
  })

  it('calcula IVA (15%) correctamente', () => {
    const products = [{ quantity: 10, price: 10.00 }]
    const result = calculateOrderTotals(products)
    expect(result.taxes).toBe(15.00)
  })
})
\end{lstlisting}

\newpage

\subsection{Modulo de Mezclas Personalizadas}

\subsubsection{Descripcion Funcional}
El modulo permite a los usuarios crear mezclas personalizadas de frutos secos, con calculo automatico de precios e informacion nutricional.

\subsubsection{Casos de Prueba}

\begin{table}[H]
\centering
\caption{Casos de prueba - Mezclas Personalizadas}
\label{tab:cp-mezclas}
\begin{tabular}{|c|p{6cm}|c|c|}
\hline
\textbf{ID} & \textbf{Descripcion} & \textbf{Tipo} & \textbf{Estado} \\
\hline
CP-M01 & Validacion de nombre (3-25 caracteres) & Validacion & $\checkmark$ \\
\hline
CP-M02 & Rechazo de caracteres especiales en nombre & Validacion & $\checkmark$ \\
\hline
CP-M03 & Validacion de cantidad positiva & Validacion & $\checkmark$ \\
\hline
CP-M04 & Validacion de stock disponible & Validacion & $\checkmark$ \\
\hline
CP-M05 & Calculo de precio por componente & Calculo & $\checkmark$ \\
\hline
CP-M06 & Calculo de precio total de mezcla & Calculo & $\checkmark$ \\
\hline
CP-M07 & Calculo de calorias totales & Calculo & $\checkmark$ \\
\hline
CP-M08 & Calculo de macronutrientes & Calculo & $\checkmark$ \\
\hline
CP-M09 & Agregar componente a mezcla & Funcional & $\checkmark$ \\
\hline
CP-M10 & Eliminar componente de mezcla & Funcional & $\checkmark$ \\
\hline
CP-M11 & Actualizar cantidad de componente & Funcional & $\checkmark$ \\
\hline
CP-M12 & Persistencia de mezcla guardada & Persistencia & $\checkmark$ \\
\hline
\end{tabular}
\end{table}

\subsubsection{Codigo de Prueba - Calculos Nutricionales}

\begin{lstlisting}[language=JavaScript, caption=Test de calculos nutricionales]
describe('Calculos nutricionales', () => {
  const nutritionalData = {
    'A01': { calories: 579, protein: 21.2, fat: 49.9, carbs: 21.6 },
    'N01': { calories: 654, protein: 15.2, fat: 65.2, carbs: 13.7 },
    'P01': { calories: 299, protein: 3.1, fat: 0.5, carbs: 79.2 },
  }

  const calculateMixNutrition = (components) => {
    const totals = { calories: 0, protein: 0, fat: 0, carbs: 0 }

    components.forEach(component => {
      const nutrition = nutritionalData[component.productCode]
      if (nutrition) {
        const factor = component.quantity
        totals.calories += nutrition.calories * factor
        totals.protein += nutrition.protein * factor
        totals.fat += nutrition.fat * factor
        totals.carbs += nutrition.carbs * factor
      }
    })
    return totals
  }

  it('calcula calorias totales correctamente', () => {
    const components = [{ productCode: 'A01', quantity: 1 }]
    const nutrition = calculateMixNutrition(components)
    expect(nutrition.calories).toBe(579)
  })

  it('suma nutrientes de multiples componentes', () => {
    const components = [
      { productCode: 'A01', quantity: 1 },
      { productCode: 'P01', quantity: 1 }
    ]
    const nutrition = calculateMixNutrition(components)
    expect(nutrition.calories).toBe(878) // 579 + 299
  })
})
\end{lstlisting}

\newpage

\subsection{Modulo de Funciones Utilitarias}

\subsubsection{Casos de Prueba}

\begin{table}[H]
\centering
\caption{Casos de prueba - Funciones Utilitarias}
\label{tab:cp-utils}
\begin{tabular}{|c|p{5cm}|p{3cm}|p{3cm}|c|}
\hline
\textbf{ID} & \textbf{Funcion} & \textbf{Entrada} & \textbf{Salida} & \textbf{Estado} \\
\hline
CP-U01 & formatDateToDDMMYYYY & Date(2026,0,15) & 15/01/2026 & $\checkmark$ \\
\hline
CP-U02 & parseDDMMYYYYToDate & 15/01/2026 & Date object & $\checkmark$ \\
\hline
CP-U03 & formatCurrency & 1234.567 & 1,234.57 & $\checkmark$ \\
\hline
CP-U04 & isEmpty & "" & true & $\checkmark$ \\
\hline
CP-U05 & isEmpty & null & true & $\checkmark$ \\
\hline
CP-U06 & isEmpty & [] & true & $\checkmark$ \\
\hline
CP-U07 & isValidNumber & 50, [0,100] & true & $\checkmark$ \\
\hline
CP-U08 & normalizeText & "Cafe" & "cafe" & $\checkmark$ \\
\hline
CP-U09 & generateUniqueId & "PRD" & PRD-xxxxx & $\checkmark$ \\
\hline
CP-U10 & deepEqual & \{a:1\}, \{a:1\} & true & $\checkmark$ \\
\hline
\end{tabular}
\end{table}

\newpage

% ============================================
% SECCION 5: EJECUCION DE PRUEBAS
% ============================================
\section{Ejecucion de Pruebas}

\subsection{Procedimiento de Ejecucion}

Para ejecutar las pruebas unitarias, se deben seguir los siguientes pasos:

\begin{enumerate}
    \item \textbf{Preparacion del entorno:}
    \begin{lstlisting}[language=bash]
# Navegar al directorio del proyecto
cd KairosMix_V2.0

# Instalar dependencias
npm install
    \end{lstlisting}
    
    \item \textbf{Ejecucion de pruebas:}
    \begin{lstlisting}[language=bash]
# Ejecutar todas las pruebas (una vez)
npm run test:run

# Ejecutar en modo watch (desarrollo)
npm test

# Ejecutar con interfaz grafica
npm run test:ui
    \end{lstlisting}
    
    \item \textbf{Generacion de reporte de cobertura:}
    \begin{lstlisting}[language=bash]
# Generar reporte de cobertura
npm run test:coverage
    \end{lstlisting}
\end{enumerate}

\subsection{Comandos Disponibles}

\begin{table}[H]
\centering
\caption{Referencia de comandos de testing}
\label{tab:comandos}
\begin{tabular}{|l|p{8cm}|}
\hline
\textbf{Comando} & \textbf{Descripcion} \\
\hline
\comando{npm run test:run} & Ejecuta todas las pruebas una sola vez \\
\hline
\comando{npm test} & Ejecuta pruebas en modo watch (desarrollo) \\
\hline
\comando{npm run test:ui} & Abre interfaz grafica interactiva de Vitest \\
\hline
\comando{npm run test:coverage} & Genera reporte detallado de cobertura \\
\hline
\end{tabular}
\end{table}

\newpage

% ============================================
% SECCION 6: RESULTADOS
% ============================================
\section{Resultados de la Ejecucion}

\subsection{Resumen Ejecutivo}

\begin{resultbox}[Resultado General: EXITOSO]
\begin{center}
\Large\textbf{176 pruebas ejecutadas - 100\% exitosas}
\end{center}

\vspace{0.5cm}

\begin{tabular}{ll}
\textbf{Archivos de test:} & 6 archivos \\
\textbf{Tests ejecutados:} & 176 tests \\
\textbf{Tests exitosos:} & 176 (100\%) \\
\textbf{Tests fallidos:} & 0 (0\%) \\
\textbf{Tiempo total:} & 3.66 segundos \\
\end{tabular}
\end{resultbox}

\subsection{Resultados por Modulo}

\begin{table}[H]
\centering
\caption{Resultados detallados por archivo de prueba}
\label{tab:resultados}
\begin{tabular}{|l|c|c|c|c|}
\hline
\textbf{Archivo} & \textbf{Tests} & \textbf{Exitosos} & \textbf{Fallidos} & \textbf{Tiempo} \\
\hline
seedData.test.js & 20 & 20 & 0 & 19ms \\
\hline
utils.test.js & 33 & 33 & 0 & 49ms \\
\hline
CustomMixDesigner.test.jsx & 32 & 32 & 0 & 12ms \\
\hline
OrderManager.test.jsx & 35 & 35 & 0 & 14ms \\
\hline
ClientManager.test.jsx & 34 & 34 & 0 & 275ms \\
\hline
ProductManager.test.jsx & 22 & 22 & 0 & 1277ms \\
\hline
\textbf{TOTAL} & \textbf{176} & \textbf{176} & \textbf{0} & \textbf{3.66s} \\
\hline
\end{tabular}
\end{table}

\subsection{Evidencia de Ejecucion}

\begin{figure}[H]
\centering
\begin{verbatim}
> kairosmix@0.0.0 test:run
> vitest run

 RUN  v4.0.17 D:/Semestre VII/.../KairosMix_V2.0

 [PASS] src/data/seedData.test.js (20 tests) 19ms
 [PASS] src/utils/utils.test.js (33 tests) 49ms
 [PASS] src/components/CustomMix/CustomMixDesigner.test.jsx (32 tests) 12ms
 [PASS] src/components/Orders/OrderManager.test.jsx (35 tests) 14ms
 [PASS] src/components/Clients/ClientManager.test.jsx (34 tests) 275ms
 [PASS] src/components/Products/ProductManager.test.jsx (22 tests) 1277ms

 Test Files  6 passed (6)
      Tests  176 passed (176)
   Start at  12:29:32
   Duration  3.66s
\end{verbatim}
\caption{Salida de consola de la ejecucion de pruebas}
\label{fig:evidencia}
\end{figure}

\subsection{Analisis de Rendimiento}

\begin{table}[H]
\centering
\caption{Metricas de rendimiento de las pruebas}
\label{tab:rendimiento}
\begin{tabular}{|l|c|c|c|}
\hline
\textbf{Archivo} & \textbf{Tiempo (ms)} & \textbf{Tests/seg} & \textbf{Observacion} \\
\hline
seedData.test.js & 19 & 1,052 & Pruebas de datos estaticos \\
\hline
CustomMixDesigner.test.jsx & 12 & 2,667 & Logica pura sin DOM \\
\hline
OrderManager.test.jsx & 14 & 2,500 & Logica de negocio \\
\hline
utils.test.js & 49 & 673 & Funciones utilitarias \\
\hline
ClientManager.test.jsx & 275 & 124 & Componentes React \\
\hline
ProductManager.test.jsx & 1,277 & 17 & UI compleja + async \\
\hline
\end{tabular}
\end{table}

\begin{notabox}[Observacion sobre tiempos de ejecucion]
Los tests de \archivo{ProductManager.test.jsx} presentan mayor tiempo de ejecucion debido a:
\begin{itemize}
    \item Renderizado completo de componentes React con estado
    \item Simulacion de eventos de usuario (typing, clicks, uploads)
    \item Operaciones asincronas con \codigofuente{waitFor}
    \item Validaciones en tiempo real del formulario
\end{itemize}
\end{notabox}

\newpage

% ============================================
% SECCION 7: ANALISIS Y CONCLUSIONES
% ============================================
\section{Analisis y Conclusiones}

\subsection{Analisis de Resultados}

\subsubsection{Cobertura Funcional}
Se logro una cobertura del 100\% en todas las funcionalidades criticas del sistema:

\begin{itemize}
    \item \textbf{Validaciones de entrada:} Todas las reglas de validacion para formularios fueron verificadas, incluyendo documentos de identidad ecuatorianos, formatos de email y telefono.
    
    \item \textbf{Logica de negocio:} Los calculos de precios, totales con IVA (15\%), e informacion nutricional funcionan correctamente.
    
    \item \textbf{Maquina de estados:} Las transiciones de estado de pedidos siguen el flujo definido, previniendo transiciones invalidas.
    
    \item \textbf{Generacion de identificadores:} Los codigos de producto se generan automaticamente siguiendo el patron establecido.
\end{itemize}

\subsubsection{Calidad del Codigo}
La implementacion de pruebas unitarias revelo:

\begin{itemize}
    \item Codigo modular y desacoplado que facilita el testing
    \item Separacion clara entre logica de negocio y presentacion
    \item Manejo adecuado de casos limite y valores nulos
\end{itemize}

\subsection{Conclusiones}

\begin{enumerate}
    \item \textbf{Objetivo cumplido:} Se implementaron 176 pruebas unitarias con una tasa de exito del 100\%, cumpliendo con el objetivo de verificar el correcto funcionamiento del sistema.
    
    \item \textbf{Validaciones robustas:} Las validaciones de datos de entrada, especialmente para documentos ecuatorianos (cedula, RUC, pasaporte), funcionan correctamente.
    
    \item \textbf{Logica de negocio verificada:} Los calculos financieros y nutricionales producen resultados precisos y consistentes.
    
    \item \textbf{Flujos de trabajo validados:} La maquina de estados de pedidos opera segun las especificaciones, evitando transiciones invalidas.
    
    \item \textbf{Base para integracion continua:} Las pruebas estan listas para integrarse en un pipeline CI/CD.
\end{enumerate}

\subsection{Recomendaciones}

\begin{alertbox}[Mejoras Futuras Sugeridas]
\begin{enumerate}
    \item \textbf{Pruebas de Integracion:} Implementar tests que verifiquen la interaccion entre modulos.
    
    \item \textbf{Pruebas End-to-End:} Considerar Playwright o Cypress para simular flujos completos de usuario.
    
    \item \textbf{Cobertura de Codigo:} Ejecutar analisis de cobertura periodicamente con \comando{npm run test:coverage}.
    
    \item \textbf{Pipeline CI/CD:} Integrar las pruebas en GitHub Actions o similar para ejecucion automatica.
    
    \item \textbf{Pruebas de Rendimiento:} Evaluar tiempos de respuesta bajo carga.
\end{enumerate}
\end{alertbox}

\newpage

% ============================================
% APENDICES
% ============================================
\appendix

\section{Apendice A: Datos de Prueba (Mocks)}

\subsection{Productos de Prueba}

\begin{lstlisting}[language=JavaScript, caption=Datos mock de productos]
export const mockProducts = [
  {
    id: 1,
    code: 'A01',
    name: 'Almendras Premium',
    countryOfOrigin: 'Estados Unidos',
    pricePerPound: 15.99,
    wholesalePrice: 14.50,
    retailPrice: 17.99,
    initialStock: 50,
    stock: 50,
    image: null
  },
  {
    id: 2,
    code: 'N01',
    name: 'Nueces de Castilla',
    countryOfOrigin: 'Chile',
    pricePerPound: 22.50,
    wholesalePrice: 20.00,
    retailPrice: 25.99,
    initialStock: 30,
    stock: 30,
    image: null
  }
]
\end{lstlisting}

\subsection{Clientes de Prueba}

\begin{lstlisting}[language=JavaScript, caption=Datos mock de clientes]
export const mockClients = [
  {
    id: 1,
    name: 'Maria Gonzalez',
    idNumber: '1234567890',
    idType: 'cedula',
    email: 'maria.gonzalez@email.com',
    phone: '0987654321',
    address: 'Av. Principal 123, Quito, Ecuador'
  },
  {
    id: 2,
    name: 'Juan Perez',
    idNumber: '0987654321098',
    idType: 'ruc',
    email: 'juan.perez@empresa.com',
    phone: '0998877665',
    address: 'Calle Secundaria 456, Guayaquil, Ecuador'
  }
]
\end{lstlisting}

\newpage

\section{Apendice B: Configuracion de Scripts en package.json}

\begin{lstlisting}[language=JavaScript, caption=Seccion de scripts del package.json]
{
  "scripts": {
    "dev": "vite",
    "build": "vite build",
    "preview": "vite preview",
    "test": "vitest",
    "test:run": "vitest run",
    "test:ui": "vitest --ui",
    "test:coverage": "vitest run --coverage"
  }
}
\end{lstlisting}

\section{Apendice C: Estructura de Archivos del Proyecto}

\begin{verbatim}
KairosMix_V2.0/
  src/
    components/
      Products/
        ProductManager.jsx
        ProductManager.test.jsx
        ProductForm.jsx
      Clients/
        ClientManager.jsx
        ClientManager.test.jsx
        ClientForm.jsx
      Orders/
        OrderManager.jsx
        OrderManager.test.jsx
        OrderForm.jsx
      CustomMix/
        CustomMixDesigner.jsx
        CustomMixDesigner.test.jsx
    utils/
      sweetAlertConfig.js
      utils.test.js
    data/
      seedData.js
      seedData.test.js
    test/
      setup.js
      testUtils.jsx
  vitest.config.js
  package.json
\end{verbatim}

\end{document}
