\documentclass[12pt,a4paper]{article}
\usepackage[utf8]{inputenc}
\usepackage[spanish]{babel}
\usepackage{geometry}
\usepackage{graphicx}
\usepackage{tabularx}
\usepackage{longtable}
\usepackage{booktabs}
\usepackage{xcolor}
\usepackage{fancyhdr}
\usepackage{hyperref}
\usepackage{float}

\geometry{margin=2.5cm}

\pagestyle{fancy}
\fancyhf{}
\fancyhead[L]{Plan de Pruebas Unitarias - Kairos Mix}
\fancyhead[R]{27837\_G6\_ADS}
\fancyfoot[C]{\thepage}

\hypersetup{
    colorlinks=true,
    linkcolor=black,
    urlcolor=cyan
}

\begin{document}

% Portada
\begin{titlepage}
    \centering
    \vspace*{2cm}
    
    {\Huge\bfseries Plan de Pruebas Unitarias\par}
    \vspace{0.5cm}
    {\Large Sistema Kairos Mix\par}
    \vspace{1cm}
    {\large Tienda de frutos secos ``Kairos de Dios''\par}
    
    \vspace{2cm}
    
    \begin{tabular}{ll}
        \textbf{Proyecto:} & Kairos Mix \\
        \textbf{Codigo ECS:} & PP \\
        \textbf{Version:} & 4.0.0 \\
        \textbf{Fecha:} & 22 de enero de 2026 \\
        \textbf{ID Proyecto:} & 27837\_G6\_ADS \\
    \end{tabular}
    
    \vfill
    
    \textbf{Equipo de Desarrollo:}\\
    \vspace{0.5cm}
    \begin{tabular}{l}
        Caetano Flores \\
        Jordan Guaman \\
        Anthony Morales \\
        Leonardo Narvaez \\
    \end{tabular}
    
    \vspace{1cm}
    {\large Enero 2026\par}
\end{titlepage}

\tableofcontents
\newpage

%-------------------------------------------------

\section{Introduccion}

\subsection{Proposito}
El presente documento define el plan y reporte de resultados de las pruebas unitarias del sistema Kairos Mix. Se detallan los módulos evaluados, los casos de prueba ejecutados y los resultados obtenidos en la validación del código fuente.

\subsection{Alcance}
Este plan se limita \textbf{exclusivamente} a la validación de componentes individuales del sistema mediante \textbf{pruebas unitarias} automatizadas. El alcance cubre los componentes de gestión de datos, utilidades, y componentes de interfaz definidos en el directorio \texttt{src}.

\subsection{Documentos de Referencia}
\begin{itemize}
    \item SRS - Especificación de Requisitos de Software
    \item Repositorio de Código Fuente (Branch v4.0.17)
\end{itemize}

%-------------------------------------------------

\section{Objetivos de las Pruebas Unitarias}

\subsection{Objetivo General}
Verificar la integridad y corrección lógica de los componentes React y funciones de utilidad JavaScript del proyecto Kairos Mix.

\subsection{Objetivos Especificos}
\begin{itemize}
    \item Validar las funciones de utilidad (formateo, validación, cálculos).
    \item Asegurar que los componentes de gestión (Productos, Clientes, Pedidos) funcionen según lo esperado.
    \item Verificar la lógica del Diseñador de Mezclas (CustomMix).
    \item Confirmar la integridad de los datos semilla (Seed Data).
\end{itemize}

%-------------------------------------------------

\section{Estrategia de Pruebas}

\subsection{Herramientas Utilizadas}
\begin{itemize}
    \item \textbf{Framework de Pruebas:} Vitest (Compatible con Jest)
    \item \textbf{Entorno:} JSDOM / Node.js
    \item \textbf{Librerías Auxiliares:} React Testing Library
\end{itemize}

\subsection{Resumen de Ejecucion}
Se han ejecutado un total de \textbf{176 pruebas unitarias} distribuidas en 6 archivos de prueba principales. El tiempo total de ejecución fue de aproximadamente 5.58 segundos.

\begin{table}[h]
\centering
\begin{tabular}{|l|c|c|}
\hline
\textbf{Archivo de Prueba} & \textbf{Modulo} & \textbf{Estado} \\
\hline
src/utils/utils.test.js & Utilidades Generales & \textcolor{green}{PASSED} (33 tests) \\
\hline
src/components/CustomMix/ & Diseñador de Mezclas & \textcolor{green}{PASSED} (32 tests) \\
CustomMixDesigner.test.jsx & & \\
\hline
src/components/Orders/ & Gestión de Pedidos & \textcolor{green}{PASSED} (35 tests) \\
OrderManager.test.jsx & & \\
\hline
src/components/Clients/ & Gestión de Clientes & \textcolor{green}{PASSED} (34 tests) \\
ClientManager.test.jsx & & \\
\hline
src/components/Products/ & Gestión de Productos & \textcolor{green}{PASSED} (22 tests) \\
ProductManager.test.jsx & & \\
\hline
src/data/seedData.test.js & Datos Semilla & \textcolor{green}{PASSED} (20 tests) \\
\hline
\end{tabular}
\caption{Resumen de Archivos Probados}
\end{table}

%-------------------------------------------------

\section{Resultados de Pruebas por Modulo}

\subsection{Modulo: Utilidades (src/utils/utils.test.js)}
Este módulo contiene funciones críticas transversales a toda la aplicación. Se validaron 33 casos de prueba con éxito.

\begin{longtable}{|p{6cm}|p{7cm}|}
\hline
\textbf{Categoría} & \textbf{Validaciones Realizadas} \\
\hline
Formateo de fechas & Formato DD/MM/YYYY, agregación de ceros, parseo inverso. \\
\hline
Formateo de moneda & Decimales (2), redondeo, separadores de miles. \\
\hline
Validaciones comunes & Strings vacíos, null/undefined, arrays vacíos, validación de números y rangos. \\
\hline
Generación de IDs & Prefijos, unicidad, integridad del timestamp. \\
\hline
Normalización de texto & Conversión a minúsculas, remoción de acentos y espacios. \\
\hline
Comparación de objetos & Igualdad profunda, diferencias, manejo de primitivos y nulos. \\
\hline
Cálculos de porcentajes & Cálculo exacto, manejo de ceros. \\
\hline
Configuración SweetAlert & Diálogos de confirmación y eliminación. \\
\hline
Manejo de localStorage & Retorno por defecto, parseo JSON, manejo de errores. \\
\hline
\caption{Detalle de pruebas - Utils}
\end{longtable}

\subsection{Modulo: Gestión de Productos (ProductManager)}
Se validaron 22 casos de prueba enfocados en la administración del inventario.
\begin{itemize}
    \item Renderizado correcto del listado de productos.
    \item Funcionalidad de agregar nuevo producto.
    \item Edición de información de productos existentes.
    \item Eliminación lógica o física de items.
    \item Validación de campos en formularios de producto.
\end{itemize}

\subsection{Modulo: Diseñador de Mezclas (CustomMixDesigner)}
Componente crítico para el valor del negocio. Se ejecutaron 32 pruebas.
\begin{itemize}
    \item Selección de ingredientes base.
    \item Cálculo dinámico de información nutricional.
    \item Restricciones de mezcla (porcentajes, pesos).
    \item Visualización de resumen de mezcla.
\end{itemize}

\subsection{Modulo: Gestión de Clientes (ClientManager)}
Se verificaron 34 casos relacionados con la cartera de clientes.
\begin{itemize}
    \item CRUD de clientes.
    \item Búsqueda y filtrado de usuarios.
    \item Validación de datos de contacto (email, teléfono).
    \item Historial de pedidos por cliente.
\end{itemize}

\subsection{Modulo: Gestión de Pedidos (OrderManager)}
Se ejecutaron 35 pruebas para asegurar el flujo de ventas.
\begin{itemize}
    \item Creación de nuevos pedidos.
    \item Cambios de estado (Pendiente $\rightarrow$ Completado).
    \item Cálculo de totales y subtotales.
    \item Asociación de clientes y productos a la orden.
\end{itemize}

\subsection{Modulo: Datos Semilla (seedData.test.js)}
Se aseguraron 20 pruebas para verificar la integridad de los datos iniciales.
\begin{itemize}
    \item Estructura correcta del JSON inicial.
    \item Existencia de categorías y productos base requeridos.
    \item Consistencia de IDs y relaciones en datos precargados.
\end{itemize}

%-------------------------------------------------

\section{Evidencias de Ejecucion}

A continuación se presentan las capturas de pantalla obtenidas durante la ejecución de las pruebas unitarias.

\subsection{Ejecucion Exitosa de Suite Completa}
La siguiente imagen muestra la ejecución completa de todos los archivos de prueba en modo UI de Vitest, confirmando que las 176 pruebas pasaron satisfactoriamente.

\begin{figure}[H]
    \centering
    \includegraphics[width=0.95\textwidth]{imagenes/Ejecución Exitosa de Suite Completa (Modo UI).jpeg}
    \caption{Ejecución exitosa de la suite completa de pruebas en Vitest UI}
\end{figure}

\subsection{Verificacion Unitaria - Modulo Utils}
Esta captura detalla los 33 casos de prueba del módulo de utilidades, mostrando todas las categorías validadas: formateo de fechas, moneda, validaciones, generación de IDs, normalización de texto, comparación de objetos, cálculos de porcentajes y manejo de localStorage.

\begin{figure}[H]
    \centering
    \includegraphics[width=0.95\textwidth]{imagenes/Verificación Unitaria - Módulo Utils (Post-Fix).jpeg}
    \caption{Detalle de pruebas unitarias del módulo Utils - Todas pasadas}
\end{figure}

\subsection{Deteccion de Error Controlado}
Se evidencia el manejo correcto de errores en el módulo Utils, específicamente en la función \texttt{safeGetFromStorage} que maneja correctamente JSON inválido sin interrumpir la ejecución.

\begin{figure}[H]
    \centering
    \includegraphics[width=0.95\textwidth]{imagenes/Detección de Error Controlado en Utils.jpeg}
    \caption{Detección y manejo controlado de errores en localStorage}
\end{figure}

\subsection{Validacion de Integridad de Datos Semilla}
Captura que muestra la verificación de los datos iniciales del sistema (seedData), asegurando la estructura correcta de productos, categorías y relaciones.

\begin{figure}[H]
    \centering
    \includegraphics[width=0.95\textwidth]{imagenes/Validación de Integridad de Datos Semilla.jpeg}
    \caption{Pruebas de integridad de datos semilla (seedData)}
\end{figure}

%-------------------------------------------------

\section{Conclusiones}

La ejecución de pruebas unitarias para la versión actual del sistema Kairos Mix ha sido exitosa.
\begin{itemize}
    \item \textbf{Total de Pruebas:} 176
    \item \textbf{Pruebas Exitosas:} 176 (100\%)
    \item \textbf{Pruebas Fallidas:} 0
\end{itemize}

El módulo de utilidades ha demostrado robustez en el manejo de tipos de datos y formatos, lo cual es fundamental para la integridad de los datos en los módulos de negocio (Productos, Pedidos, Clientes). La cobertura de pruebas abarca los flujos principales de la aplicación.

\vfill

\begin{center}
\begin{tabular}{ccc}
\rule{5cm}{0.5pt} & \hspace{1cm} & \rule{5cm}{0.5pt} \\
Caetano Flores & & Jordan Guaman \\
Lider de Pruebas & & Tester Backend/Frontend \\
\\[1cm]
\rule{5cm}{0.5pt} & \hspace{1cm} & \rule{5cm}{0.5pt} \\
Anthony Morales & & Leonardo Narvaez \\
Tester Backend/Frontend & & Tester Backend/Frontend \\
\end{tabular}

\end{center}

\end{document}
