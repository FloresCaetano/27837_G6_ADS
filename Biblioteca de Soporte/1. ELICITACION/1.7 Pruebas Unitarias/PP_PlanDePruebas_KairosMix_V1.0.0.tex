\documentclass[12pt,a4paper]{article}
\usepackage[utf8]{inputenc}
\usepackage[spanish]{babel}
\usepackage{geometry}
\usepackage{graphicx}
\usepackage{tabularx}
\usepackage{longtable}
\usepackage{booktabs}
\usepackage{xcolor}
\usepackage{fancyhdr}
\usepackage{hyperref}

\geometry{margin=2.5cm}

\pagestyle{fancy}
\fancyhf{}
\fancyhead[L]{Plan de Pruebas - Kairos Mix}
\fancyhead[R]{27837\_G6\_ADS}
\fancyfoot[C]{\thepage}

\hypersetup{
    colorlinks=true,
    linkcolor=blue,
    filecolor=magenta,      
    urlcolor=cyan,
}

\begin{document}

% Portada
\begin{titlepage}
    \centering
    \vspace*{2cm}
    
    {\Huge\bfseries Plan de Pruebas\par}
    \vspace{0.5cm}
    {\Large Sistema Kairos Mix\par}
    \vspace{1cm}
    {\large Tienda de frutos secos "Kairos de Dios"\par}
    
    \vspace{2cm}
    
    \begin{tabular}{ll}
        \textbf{Proyecto:} & Kairos Mix \\
        \textbf{Codigo ECS:} & PP \\
        \textbf{Version:} & 1.0.0 \\
        \textbf{Fecha:} & 14 de enero de 2026 \\
        \textbf{ID Proyecto:} & 27837\_G6\_ADS \\
    \end{tabular}
    
    \vfill
    
    \textbf{Equipo de Desarrollo:}\\
    \vspace{0.5cm}
    \begin{tabular}{l}
        Caetano Flores \\
        Jordan Guaman \\
        Anthony Morales \\
        Leonardo Narvaez \\
    \end{tabular}
    
    \vspace{1cm}
    {\large Enero 2026\par}
\end{titlepage}

\tableofcontents
\newpage

\section{Introduccion}

\subsection{Proposito}
Este documento define el plan de pruebas para el sistema Kairos Mix, un aplicativo web diseñado para digitalizar los procesos de venta, gestion de inventario y pedidos de la tienda de frutos secos "Kairos de Dios", con un diferenciador competitivo: el diseñador de mezclas personalizadas con calculo nutricional automatico.

\subsection{Alcance}
El plan de pruebas abarca:
\begin{itemize}
    \item Pruebas unitarias de componentes individuales
    \item Pruebas de integracion entre capas del sistema
    \item Pruebas funcionales de casos de uso
    \item Pruebas del diferenciador: NutricionalService
    \item Pruebas de seguridad y auditoria (LogObserver)
\end{itemize}

\subsection{Documentos de Referencia}
\begin{itemize}
    \item SRS - Especificacion de Requisitos de Software (G6\_SRS\_KairosMix.pdf)
    \item DCD - Diagrama de Clases de Diseño (ClassDiagKairo\_G6\_V2.pdf)
    \item ECU - Especificacion de Casos de Uso (G6\_CasosDeUso.pdf)
    \item DAS - Descripcion de Arquitectura (Arquitectura\_G6\_V2.pdf)
\end{itemize}

\section{Objetivos de las Pruebas}

\subsection{Objetivos Generales}
\begin{enumerate}
    \item Verificar que todos los requisitos funcionales estan correctamente implementados
    \item Validar la precision del algoritmo de calculo nutricional (componente critico)
    \item Garantizar la integridad de los datos en todas las operaciones CRUD
    \item Asegurar el correcto funcionamiento del patron Observer para auditoria
    \item Validar la comunicacion entre las 3 capas del sistema
\end{enumerate}

\subsection{Criterios de Exito}
\begin{itemize}
    \item Cobertura de codigo $\geq$ 80\%
    \item 0 defectos criticos en produccion
    \item Precision del calculo nutricional: 100\% exactitud
    \item Todos los casos de prueba de alta prioridad ejecutados exitosamente
    \item Registro correcto de logs para operaciones criticas
\end{itemize}

\section{Estrategia de Pruebas}

\subsection{Niveles de Prueba}

\subsubsection{Pruebas Unitarias}
\textbf{Objetivo:} Validar componentes individuales de forma aislada.

\textbf{Herramientas:}
\begin{itemize}
    \item Backend (Node.js): Jest / Mocha
    \item Frontend (React): Jest + React Testing Library
\end{itemize}

\textbf{Responsable:} Cada desarrollador prueba sus propios componentes.

\subsubsection{Pruebas de Integracion}
\textbf{Objetivo:} Verificar la comunicacion entre modulos y capas.

\textbf{Enfoque:} Pruebas de API REST, integracion con base de datos MySQL.

\subsubsection{Pruebas Funcionales}
\textbf{Objetivo:} Validar que el sistema cumple con los casos de uso definidos.

\textbf{Basadas en:} Especificacion de Casos de Uso (ECU).

\subsection{Tipos de Prueba}

\begin{table}[h]
\centering
\begin{tabular}{|l|l|p{6cm}|}
\hline
\textbf{Tipo} & \textbf{Nivel} & \textbf{Descripcion} \\
\hline
Funcionales & Unitaria & Validacion de logica de negocio \\
\hline
Integracion & Integracion & Comunicacion entre capas \\
\hline
Regresion & Unitaria/Integracion & Verificar que nuevos cambios no rompen funcionalidad existente \\
\hline
Seguridad & Unitaria & Validacion de entradas, registro de logs \\
\hline
Performance & Sistema & Tiempo de respuesta del calculo nutricional \\
\hline
\end{tabular}
\caption{Tipos de prueba a ejecutar}
\end{table}

\section{Casos de Prueba}

\subsection{Modulo: ProductoController}

\begin{longtable}{|p{2.5cm}|p{4cm}|p{4cm}|p{3cm}|}
\hline
\textbf{ID} & \textbf{Descripcion} & \textbf{Entrada} & \textbf{Resultado Esperado} \\
\hline
\endfirsthead
\hline
\textbf{ID} & \textbf{Descripcion} & \textbf{Entrada} & \textbf{Resultado Esperado} \\
\hline
\endhead

TC-PR-001 & Crear producto con datos validos & nombre, categoria, precio, stock & Producto creado, status 201 \\
\hline
TC-PR-002 & Crear producto sin nombre & categoria, precio, stock & Error de validacion, status 400 \\
\hline
TC-PR-003 & Actualizar precio de producto & idProducto, nuevoPrecio & Precio actualizado correctamente \\
\hline
TC-PR-004 & Eliminar producto existente & idProducto & Producto eliminado, status 200 \\
\hline
TC-PR-005 & Consultar stock de producto & idProducto & Valor de stock correcto \\
\hline
\caption{Casos de prueba - ProductoController}
\end{longtable}

\subsection{Modulo: NutricionalService (CRITICO)}

\begin{longtable}{|p{2.5cm}|p{4cm}|p{4cm}|p{3cm}|}
\hline
\textbf{ID} & \textbf{Descripcion} & \textbf{Entrada} & \textbf{Resultado Esperado} \\
\hline
\endfirsthead
\hline
\textbf{ID} & \textbf{Descripcion} & \textbf{Entrada} & \textbf{Resultado Esperado} \\
\hline
\endhead

TC-NS-001 & Calcular calorias totales de mezcla & ingredientes con valores nutricionales, proporciones & Suma exacta de calorias \\
\hline
TC-NS-002 & Calcular proteinas basadas en proporciones & 50\% Almendras (20g prot), 50\% Nueces (15g prot) & 17.5g proteinas totales \\
\hline
TC-NS-003 & Calcular precio total de mezcla & Ingrediente A (\$10, 40\%), Ingrediente B (\$15, 60\%) & \$13.00 total \\
\hline
TC-NS-004 & Validar suma de porcentajes = 100\% & proporciones: 30\%, 40\%, 20\% & Error: suma != 100\% \\
\hline
TC-NS-005 & Calcular grasas de mezcla compleja & 3 ingredientes con diferentes valores & Calculo correcto ponderado \\
\hline
TC-NS-006 & Calcular carbohidratos & ingredientes, proporciones & Suma ponderada correcta \\
\hline
TC-NS-007 & Verificar precision decimal & valores con decimales & Precision hasta 2 decimales \\
\hline
\caption{Casos de prueba - NutricionalService (COMPONENTE CRITICO)}
\end{longtable}

\subsection{Modulo: PedidoController}

\begin{longtable}{|p{2.5cm}|p{4cm}|p{4cm}|p{3cm}|}
\hline
\textbf{ID} & \textbf{Descripcion} & \textbf{Entrada} & \textbf{Resultado Esperado} \\
\hline
\endfirsthead
\hline
\textbf{ID} & \textbf{Descripcion} & \textbf{Entrada} & \textbf{Resultado Esperado} \\
\hline
\endhead

TC-PE-001 & Crear pedido con estado inicial & cliente, productos & Estado = "Por preparar" \\
\hline
TC-PE-002 & Transicion valida de estados & estado actual: "Por preparar", nuevo: "Preparado" & Cambio exitoso \\
\hline
TC-PE-003 & Transicion invalida de estados & estado actual: "Entregado", nuevo: "Por preparar" & Error: transicion no permitida \\
\hline
TC-PE-004 & Calcular total del pedido & lista de productos con precios & Suma correcta del total \\
\hline
TC-PE-005 & Generar reporte de pedido & idPedido & Reporte con datos completos \\
\hline
\caption{Casos de prueba - PedidoController}
\end{longtable}

\subsection{Modulo: MezclaController}

\begin{longtable}{|p{2.5cm}|p{4cm}|p{4cm}|p{3cm}|}
\hline
\textbf{ID} & \textbf{Descripcion} & \textbf{Entrada} & \textbf{Resultado Esperado} \\
\hline
\endfirsthead
\hline
\textbf{ID} & \textbf{Descripcion} & \textbf{Entrada} & \textbf{Resultado Esperado} \\
\hline
\endhead

TC-MZ-001 & Crear mezcla personalizada & nombre, ingredientes, proporciones & Mezcla creada \\
\hline
TC-MZ-002 & Agregar ingrediente a mezcla & idMezcla, idIngrediente, proporcion & Ingrediente agregado \\
\hline
TC-MZ-003 & Invocar calculo nutricional & idMezcla & Valores nutricionales calculados \\
\hline
TC-MZ-004 & Guardar mezcla como pedido & idMezcla, idCliente & Pedido creado desde mezcla \\
\hline
\caption{Casos de prueba - MezclaController}
\end{longtable}

\subsection{Modulo: LogObserver (Patron Observer)}

\begin{longtable}{|p{2.5cm}|p{4cm}|p{4cm}|p{3cm}|}
\hline
\textbf{ID} & \textbf{Descripcion} & \textbf{Entrada} & \textbf{Resultado Esperado} \\
\hline
\endfirsthead
\hline
\textbf{ID} & \textbf{Descripcion} & \textbf{Entrada} & \textbf{Resultado Esperado} \\
\hline
\endhead

TC-LO-001 & Registrar creacion de producto & ProductoController.crear() & Log generado con timestamp \\
\hline
TC-LO-002 & Registrar cambio de estado de pedido & PedidoController.cambiarEstado() & Log con estado anterior y nuevo \\
\hline
TC-LO-003 & Registrar eliminacion de producto & ProductoController.eliminar() & Log con datos del producto eliminado \\
\hline
TC-LO-004 & Verificar formato de log & cualquier operacion critica & JSON con entidad, operacion, usuario, timestamp \\
\hline
\caption{Casos de prueba - LogObserver}
\end{longtable}

\subsection{Frontend: Componentes React}

\begin{longtable}{|p{2.5cm}|p{4cm}|p{4cm}|p{3cm}|}
\hline
\textbf{ID} & \textbf{Descripcion} & \textbf{Entrada} & \textbf{Resultado Esperado} \\
\hline
\endfirsthead
\hline
\textbf{ID} & \textbf{Descripcion} & \textbf{Entrada} & \textbf{Resultado Esperado} \\
\hline
\endhead

TC-UI-001 & Renderizar lista de productos & array de productos & Productos mostrados en pantalla \\
\hline
TC-UI-002 & Validar formulario de producto & campos vacios & Mensajes de error visibles \\
\hline
TC-UI-003 & Diseñador de mezclas - agregar ingrediente & click en ingrediente & Ingrediente aparece en lista de seleccion \\
\hline
TC-UI-004 & Mostrar valores nutricionales & mezcla calculada & Tabla nutricional visible \\
\hline
TC-UI-005 & Cambiar estado de pedido & click en boton de estado & Estado actualizado en UI \\
\hline
\caption{Casos de prueba - Componentes Frontend}
\end{longtable}

\section{Priorizacion de Casos de Prueba}

\subsection{Prioridad Critica}
\begin{itemize}
    \item \textbf{TC-NS-001 a TC-NS-007:} Calculo nutricional (diferenciador del sistema)
    \item \textbf{TC-PR-001:} Crear producto
    \item \textbf{TC-PE-001, TC-PE-002:} Gestion de pedidos y estados
    \item \textbf{TC-LO-001 a TC-LO-004:} Auditoria y seguridad
\end{itemize}

\subsection{Prioridad Alta}
\begin{itemize}
    \item \textbf{TC-MZ-001 a TC-MZ-004:} Diseñador de mezclas
    \item \textbf{TC-PR-003, TC-PR-005:} Operaciones CRUD de productos
    \item \textbf{TC-UI-003, TC-UI-004:} UI del diseñador de mezclas
\end{itemize}

\subsection{Prioridad Media}
\begin{itemize}
    \item \textbf{TC-PR-002, TC-PR-004:} Validaciones y eliminaciones
    \item \textbf{TC-PE-003, TC-PE-004, TC-PE-005:} Funciones adicionales de pedidos
    \item \textbf{TC-UI-001, TC-UI-002, TC-UI-005:} Componentes UI generales
\end{itemize}

\section{Recursos y Responsabilidades}

\subsection{Equipo de Pruebas}
\begin{table}[h]
\centering
\begin{tabular}{|l|l|}
\hline
\textbf{Rol} & \textbf{Responsable} \\
\hline
Lider de Pruebas & Caetano Flores \\
\hline
Tester Backend & Jordan Guaman \\
\hline
Tester Frontend & Anthony Morales \\
\hline
Tester de Integracion & Leonardo Narvaez \\
\hline
\end{tabular}
\caption{Equipo de pruebas}
\end{table}

\subsection{Entorno de Pruebas}
\begin{itemize}
    \item \textbf{Sistema Operativo:} Windows
    \item \textbf{Backend:} Node.js v18+
    \item \textbf{Frontend:} React 18+
    \item \textbf{Base de Datos:} MySQL 8.0 (instancia de prueba)
    \item \textbf{Framework de Pruebas:} Jest, Mocha, React Testing Library
    \item \textbf{CI/CD:} GitHub Actions (opcional)
\end{itemize}

\section{Cronograma de Pruebas}

\begin{table}[h]
\centering
\begin{tabular}{|l|l|l|}
\hline
\textbf{Fase} & \textbf{Duracion} & \textbf{Actividades} \\
\hline
Preparacion & Semana 1 & Configuracion de entorno, preparacion de datos \\
\hline
Pruebas Unitarias & Semana 2-3 & Ejecucion de TC-PR, TC-NS, TC-PE, TC-MZ \\
\hline
Pruebas de Integracion & Semana 4 & Pruebas de API y base de datos \\
\hline
Pruebas Funcionales & Semana 5 & Validacion de casos de uso completos \\
\hline
Pruebas de Regresion & Semana 6 & Re-ejecucion de casos criticos \\
\hline
Reporte Final & Semana 6 & Documentacion de resultados \\
\hline
\end{tabular}
\caption{Cronograma de pruebas}
\end{table}

\section{Criterios de Aceptacion}

\subsection{Criterios de Entrada}
\begin{itemize}
    \item Codigo fuente completo y desplegado en entorno de pruebas
    \item Base de datos configurada con datos de prueba
    \item Documentacion de diseño disponible (DCD, DAS, ECU)
    \item Herramientas de prueba instaladas y configuradas
\end{itemize}

\subsection{Criterios de Salida}
\begin{itemize}
    \item Todos los casos de prueba de prioridad critica ejecutados exitosamente
    \item 0 defectos criticos abiertos
    \item Cobertura de codigo $\geq$ 80\%
    \item Reporte de pruebas completo y aprobado
    \item Precision del NutricionalService validada al 100\%
\end{itemize}

\section{Gestion de Defectos}

\subsection{Clasificacion de Severidad}
\begin{table}[h]
\centering
\begin{tabular}{|l|p{8cm}|}
\hline
\textbf{Severidad} & \textbf{Descripcion} \\
\hline
Critico & Sistema no funcional, calculo nutricional incorrecto, perdida de datos \\
\hline
Alta & Funcionalidad principal afectada, workaround complicado \\
\hline
Media & Funcionalidad secundaria afectada, workaround disponible \\
\hline
Baja & Problemas cosmeticos o de UI \\
\hline
\end{tabular}
\caption{Niveles de severidad}
\end{table}

\subsection{Proceso de Reporte}
\begin{enumerate}
    \item Deteccion del defecto durante prueba
    \item Registro en sistema de seguimiento (carpeta 1.8 Reportes de errores)
    \item Asignacion a desarrollador responsable
    \item Correccion del defecto
    \item Re-prueba y verificacion
    \item Cierre del defecto
\end{enumerate}

\section{Entregables}

\begin{enumerate}
    \item Plan de Pruebas (este documento)
    \item Casos de Prueba Detallados (ECP)
    \item Scripts de prueba automatizadas
    \item Reporte de Ejecucion de Pruebas
    \item Reporte de Defectos
    \item Reporte de Cobertura de Codigo
    \item Informe Final de Pruebas
\end{enumerate}

\section{Riesgos y Mitigacion}

\begin{table}[h]
\centering
\begin{tabular}{|p{5cm}|p{5cm}|p{3cm}|}
\hline
\textbf{Riesgo} & \textbf{Impacto} & \textbf{Mitigacion} \\
\hline
Error en algoritmo de calculo nutricional & Critico - Funcionalidad principal incorrecta & Pruebas exhaustivas con datos reales, validacion matematica \\
\hline
Falta de datos de prueba & Alto - Retraso en pruebas & Preparar dataset de prueba con anticipacion \\
\hline
Cambios de requisitos durante pruebas & Medio - Re-trabajo & Congelar requisitos antes de inicio de pruebas \\
\hline
Falta de tiempo para pruebas completas & Alto - Pruebas incompletas & Priorizar casos criticos primero \\
\hline
\end{tabular}
\caption{Riesgos y mitigacion}
\end{table}

\section{Conclusiones}

Este plan de pruebas asegura la calidad del sistema Kairos Mix mediante:

\begin{itemize}
    \item Cobertura exhaustiva de todos los modulos criticos
    \item Enfasis especial en el diferenciador: NutricionalService
    \item Validacion de seguridad mediante LogObserver
    \item Pruebas de integracion entre las 3 capas del sistema
    \item Proceso estructurado de gestion de defectos
\end{itemize}

La ejecucion exitosa de este plan garantizara un sistema robusto, confiable y listo para produccion.

\vfill

\begin{center}
\begin{tabular}{ccc}
\rule{5cm}{0.5pt} & \hspace{1cm} & \rule{5cm}{0.5pt} \\
Caetano Flores & & Jordan Guaman \\
Lider de Pruebas & & Tester Backend \\
\\
\\
\rule{5cm}{0.5pt} & \hspace{1cm} & \rule{5cm}{0.5pt} \\
Anthony Morales & & Leonardo Narvaez \\
Tester Frontend & & Tester de Integracion \\
\end{tabular}
\end{center}

\end{document}
